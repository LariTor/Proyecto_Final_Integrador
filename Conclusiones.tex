% Capítulo con las conclusiones.
\chapter{Conclusiones}

A lo largo del trabajo se dio un panorama general del tema de 
identificación por radiofrecuencia, en particular se estudió el 
estándar ISO/IEC 14443, en su versión de comunicación tipo A; se 
analizó en detalle la interfaz inductiva de transmisión de datos y 
energía, y se diseño e implementó en silicio un circuito integrado 
que permite enviar y recibir datos según lo indica la norma y que se 
alimenta directamente de la señal de radiofrecuencia captada por la 
antena.

El estándar ISO/IEC 14443--A define una comunicación dentro de todo 
simple que es utilizada por una gran cantidad de dispositivos 
comerciales. Para el conjunto de los dispositivos pasivos, que toman 
su alimentación de la portadora enviada por el lector, el hecho de que 
el mismo lector produzca pausas en la señal para enviar información 
es un tema que complica el diseño, ya que se debe contemplar una 
falta de energía al mismo tiempo que se recibe información. Existen 
otras formas de transmisión que no dejarían a los transponders sin 
energía, como por ejemplo una modulación de amplitud del 10\%, como 
se define en la versión B de la misma norma. Sin embargo, como se 
mostró a lo largo del trabajo una modulación ASK 100\% es muy fácil 
de demodular y además es posible almacenar energía para utilizar en 
las pausas y al mismo tiempo reducir el consumo cuando se está en una 
de ellas.

En cuanto al acoplamiento inductivo, en el capítulo 
\ref{cap:AcoplamientoInductivo} se realizó un análisis en profundidad 
del vínculo existente entre lector y transponder. Mediante un modelo 
simple se analizó la dependencia de la tensión inducida en la antena 
del transponder desde un punto de vista circuital. Para ello se 
realizó la extracción de parámetros del arreglo dado por la norma 
ISO/IEC 10373--6 junto con la antena que había sido utilizada en 
experiencias previas, lo que sirvió como punto de partida para el 
análisis a partir de datos concretos. Gracias al análisis se pudo 
diferenciar dos circuitos posibles de antena, con y sin capacidad en 
resonancia, y se pudo decidir uno por sobre el otro. 

La transmisión de datos por modulación de carga se analizó también 
desde el punto de vista circuital, utilizando para ello el arreglo de 
antenas para la medición de la amplitud de modulación dado por el 
estándar. El requisito era obtener una cierta amplitud de modulación 
a la salida del arreglo, por lo tanto se desarrolló un modelo del 
mismo que incluyó la extracción de los coeficientes de acoplamiento, 
inductancias y resistencias de la estructura. Gracias a ese modelo se 
pudo observar la amplitud de modulación obtenida variando diferentes 
tipos de carga y luego decidir el uso de modulación capacitiva. 

El diseño de la antena del transponder resultó adecuado para su uso, 
sin embargo no está optimizada su forma para obtener el mayor factor 
de acoplamiento posible. Para ello hubiese sido una buena idea 
utilizar alguna herramienta de calculo de campos tridimensionales por 
elementos finitos, de forma tal de poder mejorar el flujo concatenado.

En cuanto al sistema digital, gracias a las celdas estándar 
desarrolladas por la \emph{Oklahoma State University} se pudo 
describir el diseño en RTL y sintetizarlo e implementarlo con las 
herramientas de \emph{Synopsys Inc.}. Sin embargo, el hecho de ser un 
dispositivo de señal mixta trajo algunas complicaciones en el manejo 
de las herramientas, que no son del todo compatibles.

Uno de los puntos cruciales en el diseño fue la recepción de 
información a través del muestreo de la señal \lstinline{pause}. La 
dificultad residía en que al recibir una pausa esta debía ser 
muestreada, sin embargo en ese instante el reloj del sistema digital 
se encontraba congelado y por lo tanto todo el sistema digital estaba 
inactivo. Este problema se resolvió exitosamente mediante el agregado 
de un pequeño circuito asincrónico que retenía la señal 
\lstinline{pause} hasta que esta pudiese ser leída al retornar la 
portadora y comenzar a oscilar nuevamente el reloj.

La transmisión y recepción de las tramas con información también 
resultaron exitosas. Se pudieron enviar y recibir los 256 bytes 
posibles utilizando el conexionado de eco y de esta forma comprobar 
el funcionamiento del sistema completo de forma independiente. Las 
interfaces de entrada y salida digitales, para enviar y recibir datos, 
resultaron muy convenientes por estar diseñadas de forma simétrica, 
para ser conectadas directamente una con la otra.

La verificación del funcionamiento de los bloques digitales llevó una 
gran parte del trabajo realizado. Todos los módulos fueron probados 
por separado y luego se desarrollaron una serie de \emph{tasks} que 
permitieron enviar tramas completas a todo el conjunto. La recepción 
de datos fue verificada de forma automática a través de un 
\emph{testbench} autocontenido. Por otra parte, la verificación de la 
transmisión de datos fue realizada manualmente, analizando el diagrama 
temporal de las entradas y salidas. Completar el \emph{testbench} para 
hacerlo totalmente automatizado puede ser un gran avance para 
desarrollos futuros.

En cuanto al diseño analógico, el regulador/limitador de tensión 
funcionó, en principio, correctamente. En las mediciones realizadas se 
sabe que el nivel de campo es alto debido a la amplitud de la tensión 
inducida, sin embargo no se sabe exactamente cuál es su valor. Tampoco 
fue posible verificar el funcionamiento del regulador/limitador con el 
campo máximo dado por la norma debido a la dificultad de lograr un 
campo magnético de esa magnitud y frecuencia. Sin embargo las 
mediciones realizadas sirvieron para comprobar aunque sea que el 
circuito está operativo.

Para alimentar los distintos bloques del circuito integrado se 
utilizó la tensión rectificada por un rectificador de onda completa y 
filtrada por un capacitor MOS de gran valor trabajando en acumulación. 
La capacidad obtenida permitió superar las pausas sin inconvenientes, 
gracias también a que el consumo del bloque digital en esos instantes 
fue despreciable. La reducción del consumo se logró generando la 
señal de reloj a partir de la misma portadora. De esta forma al 
arribar una pausa el sistema digital quedaba congelado, pero su 
consumo se reducía drásticamente.

La inicialización del sistema digital a través del bloque 
\emph{Power-on Reset} funcionó correctamente, así como también el 
detector de pausas y el bloque modulador de carga.

Finalmente los resultados del dispositivo funcionando en modo eco, y 
respondiendo a los 256 bytes posibles fueron todo un éxito. Luego de 
esa prueba satisfactoria puede decirse que se tiene un transponder 
capaz de \emph{obtener su energía del campo magnético generado por el 
lector}, \emph{detectar y decodificar los símbolos recibidos} e 
\emph{identificar los datos dentro de las tramas}. Además, es capaz de 
\emph{encapsular datos en una trama estándar}, \emph{codificar los 
bits}, \emph{modularlos con la subportadora}, para luego 
\emph{enviarlos hacia el lector mediante modulación de carga}.

La descripción y los resultados del trabajo fueron publicados en el 
\emph{Congreso Argentino de Micro--nano--electrónica Tecnología y 
Aplicaciones} (EAMTA) 2014. El trabajo fue aceptado para ser publicado 
en \emph{IEEE Xplore} con número ISBN.

En cuanto a trabajos futuros, en principio la construcción de un 
lector que sea capaz de generar el campo máximo indicado en la norma 
sería un gran desafío. Por otro lado, se podría hacer una análisis más 
detallado de la antena, optimizando su diseño y dimensiones para 
mejorar el coeficiente de acoplamiento con el lector. Sino, otro trabajo 
interesante sería implementar el protocolo anti-colisión, el 
estándar completo, o bien desarrollar algún tipo de sensor con 
comunicación ISO/IEC 14443--A. Cabe destacar que el trabajo aquí 
desarrollado puede utilizarse como bloque IP (\emph{intellectual 
property}) para el desarrollo de numerosos dispositivos que utilicen 
como medio de comunicación la interfaz definida por el estándar. 
Puede tratarse de sensores autónomos, tarjetas de identificación o 
cualquier tipo de elemento que deba devolver su información cuando el 
lector la requiera.
