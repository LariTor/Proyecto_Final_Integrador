\newenvironment{abstract}{
	\cleardoublepage \null \vfill 
	\begin{center}
		\bfseries 
		\abstractname 
	\end {center}
}%
{\vfill \null}


\begin{abstract}

En el presente trabajo se comenzará realizando una breve introducción 
a los sistemas de identificación por radiofrecuencia. Luego se 
analizará detalladamente el estándar ISO/IEC 14443, enfocando el 
estudio a la interfaz de comunicación tipo A. A continuación se 
presentará el diseño de un circuito integrado que cumplirá el rol de 
\emph{transponder} y que será implementado en un proceso CMOS 
estándar de \SI{0.5}{\micro\meter}.

Para el diseño del circuito integrado se comenzará por analizar en 
profundidad el vínculo existente entre lector y transponder, lo que 
permitirá entender el proceso de traspaso de energía e información y 
se verán las distintas implementaciones posibles. Luego se 
desarrollará un modelo basado en la extracción de parámetros de la 
estructura física de las antenas, que permitirá verificar los 
resultados analíticos y realizar simulaciones mediante SPICE de 
los circuitos, estando éstos conectados a un modelo realista de la 
antena.

El circuito integrado contará con diseño analógico y digital, este 
último sintetizado a partir de código RTL. Se tratará entonces de un 
dispositivo de señal mixta por lo que se deberán compatibilizar ambos 
dominios. Se mostrará el diseño digital junto con su verificación 
funcional a nivel de compuerta y el diseño analógico con las 
simulaciones realizadas.

Finalmente se cerrará el trabajo con los detalles de la 
implementación del dispositivo en el proceso de fabricación CMOS y la 
verificación de su funcionamiento.

\end{abstract}

